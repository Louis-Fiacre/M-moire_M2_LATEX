\section*{Résumé}
\addcontentsline{toc}{chapter}{Résumé}

Cette étude propose d'étudier le pamphlet en tant que genre littéraire. Le pamphlet est un phénomène littéraire propre au XIXe siècle qui a connu une grande popularité jusqu'à la seconde guerre mondiale et qui tend depuis à se résorber voire à disparaître. Une étude générique de l'écriture pamphlétaire combinant des méthodes issues des sciences computationnelles avec des analyses stylistiques permettra d'étoffer le champ de recherche des marges de la littérature d'où ce genre est relégué. Le pamphlet comme phénomène politique et historique circonstancié a pu être étudié au travers de nombreux axes de recherche autre que littéraire. Après les travaux de Marc Angenot sur la typologie du pamphlet \textit{La Parole pamphlétaire, Contribution a la typologie des discours modernes, Payot, 1982}, nous considérons que l'apport de méthodes computationnelles pourront enrichir le regard de la recherche sur le objet littéraire qu'est le pamphlet. Nous mobiliserons pour cette étude des outils de textométrie, de linguistique et de stylistique pour analyser la spécificité de l'écriture pamphlétaire.

\medskip

\textbf{Mots-clés:} Genre; Pamphlet; Stylistique; Humanités numériques; Textométrie

\textbf{Informations bibliographiques:} Louis-Fiacre Franchet d'Espèrey, \textit{Le pamphlet, un genre littéraire ?: Étude textométrique du genre pamphlétaire}, mémoire de master 2 \og Humanités numériques et computationnelles\fg{}, dirs. Camps, Jean-Baptiste et PAILLET Anne-Marie, Université Paris, Sciences \& Lettres, 2023.

\section*{Abstract}
\addcontentsline{toc}{chapter}{Abstract}
This study undertakes an examination of the pamphlet as a literary genre. The pamphlet emerged as a distinctive literary phenomenon during the 19th century, achieving considerable popularity until the outbreak of the Second World War, subsequently undergoing a process of obsolescence. Employing a synthetic approach informed by both computational methodologies and stylistic analyses, this investigation aims to expand the scholarly purview into the peripheries of the literary domain to which the pamphlet has been relegated. In addition to its conventional literary considerations, the pamphlet, as a nuanced political and historically contingent phenomenon, has been subject to multivarious investigatory trajectories. Building upon the foundational typological framework established by Marc Angenot \textit{La Parole pamphlétaire, Contribution a la typologie des discours modernes, Payot, 1982}, we posit that the incorporation of computational methodologies stands to augment the scholarly inquiry into the distinct literary object that is the pamphlet. Leveraging tools encompassing textometric analysis, linguistic examination, we will engage in a comprehensive exploration of the distinctive qualities underpinning pamphlet composition.
\medskip

\textbf{Keywords:} Genre; Pamphlet; Stylistics; Digital Humanities; Textometry

\textbf{Bibliographic Information:} Louis-Fiacre Franchet d'Espèrey, \textit{Le pamphlet, un genre littéraire ?: Étude textométrique du genre pamphlétaire}, M.A. thesis \og Digital and computational humanities\fg{}, dirs. Camps, Jean-Baptiste et PAILLET Anne-Marie, Université Paris, Sciences \& Lettres, 2023.