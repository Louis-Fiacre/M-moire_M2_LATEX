\chapter{Conclusion générale}

\section{Conclusion}

Pour résumé le propos de cette étude, nous avons déployé différents outils en employant une démarche hypothético-déductive sur l'aspect générique du pamphlet. Notre point de départ fut basé sur le travail de Cédric Passard et de Marc Angenot qui tout deux ont contribués à renouveller les études sur le phénomène du pamphlet. Nos méthodes quantitatives ont nécéssités d'aggréger un grand nombre de texte. Pour cela nous avons constitué différents corpus de texte propre à chacun des choix méthodologiques. 

Nous avons utilisé un classifieur SVM pour détecter les occurrences de classes associables aux genres du roman, de l'essai, de la nouvelle, de l'article, des mémoires et biographies et d'un ensemble de pamphlet. Deux corpus distinct on servi l'un a comparer des textes de pamphlet face à une multitude d'autres genres d'autres auteurs, et l'autre à comparer des genres différents dont le pamphlet sur un même groupe restreint d'auteurs pamphlétaires.
Nous avons pu observer conformément à nos attentes une distinction nette entre ces classes notamment entre les genres narratifs et le pamphlet dans les deux corpus. Une distinction plus fine existe entre le genre de l'essai et le pamphlet pour notre second corpus, le seul à comporter des essais.
À partir de cela nous avons poursuivi notre recherche en nous tournant vers des méthodes non supervisées tel le \textit{clustering ascendant hiérarchique} et l'analyse en composantes principales. Cette seconde salve d'analyse fut propice à mieux cerner la proximité des textes les uns pour les autres et de pouvoir spatialiser la distance qui les séparent ou les rapprochent. La similarité autoriale expliqua beaucoup les rapprochements des oeuvres mais à part pour la comparaison du pamphlet et de l'essai, la majorité des genres se distinguèrent nettement visuellement. Nous avons vu apparaître un groupe de pamphlet se détacher fortement de l'ensemble pamphlétaire, ce qui fait de ces textes un ensemble spécieux de pamphlet qui n'en sont pas entièrement.
Ces deux approches supervisées et non supervisées on montré que si le pamphlet se distingue aisément de différents genre, l'essai sont plus proche parent dans notre corpus est assimilable voir dans le cas des analyses non-supervisées, indiscernable du pamphlet.

Nous avons tenté de développer les spécificités de l'énonciation pamphlétaire pour pouvoir quantitativement constater une distinction du pamphlet et de l'essai. Notre méthode trop rigide n'a pas permis de constater un écart significatif entre ces deux ensembles. 
Une dernière méthode de recherche de syntagme générique ou recherche de motifs à été effectué pour voir poindre des structures proche d'un style collectif des textes pamphlétaire et les premiers résultats encourageant mériteraient d'être approfondis.

En somme, cette étude bien qu'infructueuse a permis de déployer un dialogue entre des méthodes quantitatives très diverse et une réflexion linguistique et stylistique sur les spécificités génériques des textes.

\section{Ouverture}


De nombreuses réflexions laissées à l'état d'ébauche mériteraient une poursuite et un approfondissement. Notamment sur la recherche des aspects spécifiques de l'énonciation par la comparaison de l'usage des pronoms. Nous imaginons très bien une visualisation en réseaux dont les différents noeuds serait les instances locutoires et dont les arrêtes seraient la fréquence des adresses des pronoms personnels sujets vers les pronoms personnels complément d'objet indirect. Nous pourrions y constater des autoréférences comme avec enquote{vous vous trompez} et des réseaux de distribution et de flux distinct en fonction des genres.

Pour l'ébauche d'analyse des motifs, nous aurions souhaité implémenter la méthode suivi par T. Poibeau et al dans l'analyse du cliché, avec la base de l’indice statistique de l’Information Mutuelle pour extraire nos descripteurs. 
Ainsi ce travail n'est pas achevé et appelle une analyse approfondie des structures syntagmatiques pouvant avoir un impact générique plus grand que celui du signal autorial.